\label{sec:introduction}
RAVEN (Risk Analysis and Virtual control ENviroment), under the support of the Nuclear Energy Advanced Modeling and Simulation (NEAMS) program ~\cite{neams}, is increasing its capabilities to perform probabilistic analysis of stochastic dynamic systems. This supports the goal of providing the tools needed by the Risk Informed Safety Margin Characterization (RISMC) path-lead ~\cite{mandelliANS_RISMC} under the Department of Energy (DOE) Light Water Reactor Sustainability program~\cite{lwrs}. In particular, the development of RAVEN in conjunction with the thermal-hydraulic code RELAP-7~\cite{relap7FY12}, will allow the deployment of advanced methodologies for nuclear power plant (NPP) safety analysis at the industrial level. The investigation of accident scenarios in a probabilistic environment for a complex system (i.e. NPPs) is not a minor task. The complexity of such systems, and a large quantity of stochastic parameters, lead to demanding computational requirements (several CPU/hour). Moreover, high consequence scenarios are usually located in low probability regions of the input space, making even more computational demands of the risk assessment process.

This extreme need for computational power leads to the necessity to investigate methodologies for the most efficient use of available computational resources, either by increasing effectiveness of the global exploration of input space, or by focusing on regions of interest (e.g. failure/success boundaries, etc.). The milestone reported in September 2013 ~\cite{DETmilestone2013} described the capability of RAVEN to perform exploration of the uncertain domain (probabilistic space) through the support of the well-known Dynamic Event Tree (DET) approach. This report will show that the Dynamic Event Tree approach can be considered intrinsically adaptive around the failure prone input zone, if one or more of the uncertain parameters is/are responsible for the transition of interest (e.g. failure or success). Leveraging on this feature is a natural choice to extend the classical DET approach to the Adaptive Dynamic Event Tree (ADET). This extension of the DET methodology to ADET and its implementation in the RAVEN code is the subject of this report. In order to show the effectiveness of this methodology, a Station Black Out (SBO) scenario for a Pressurized Water Reactor has been employed. The ADET approach will be used to focus the exploration of the input space toward the computation of the failure probability of the system (i.e. clad failure). This report is organized in four additional sections. Section 2 recalls the concept of the DET methodology. Section 3 reports how the newer developed algorithm is employed. Section 4 is focused on the analysis performed on the PWR SBO, and, section 5 draws the conclusions.

%\subsection{subsection}
%text

%figure template

%\subsubsection{subsubsection}
%more text
%\paragraph{paragraph}
%lot of text
%\subparagraph{subparagraph}
%if you arrive at this point you have issues
